\chapter[Caracterização da Organização]{Caracterização da Organização}

O Centro de Informática da Universidade de Brasília é um Órgão Complementar da universidade, responsável pela Tecnologia da Informação e subordinado ao Decanato de Planejamento e Orçamento (DPO).  

O Centro foi criado em 1972 e iniciou suas atividades nas áreas  de apoio à pesquisa acadêmica e à informatização administrativa 
da UnB tendo como meta o desenvolvimento da informática como ferramenta de trabalho a serviço da eficiência institucional.\cite{cpd2012relatorio} 

\section*{Missão}
Viabilizar soluções de tecnologia da informação que promovam a disponibilidade, integridade, confiabilidade e autenticidade das informações dos ativos relacionados aos sistemas informatizados da Universidade de Brasília.

\section*{Visão}
Ser referência nacional, como Centro de Informática para as Universidades Federais Brasileiras e para o Ministério da Educação, reconhecido como um Centro sólido pela excelência dos produtos e serviços tecnológicos oferecidos.

\section*{Objetivos Estratégicos}

Os objetivos abaixo citados estão de acordo com o Plano Diretor de Tecnologia da Informação da Universidade de Brasília\cite{unb2014-2017pdti}, definindo os objetivos durante o interstício de 2014 à 2017.

\begin{enumerate}
\item Aprimoramento da comunicação das áreas responsáveis da TIC com a comunidade da UnB
\item Aprimorar o alinhamento, o planejamento e a organização dos serviços de TICs prestados à comunidade da UnB
\item Aprimorar  a  construção,  a  aquisição  e  a  implementação  de  Serviços  de  TICs  prestados  à comunidade da UnB
\item Aprimorar a entrega, o suporte e a operação de TICs prestados à comunidade da UnB
\item Promover atualização tecnológica dos sistemas 
e da infraestrutura de TIC da UnB
\item Garantir a conectividade, qualidade e segurança dos serviços de TICs
\item Prover serviços de qualidade de forma tempestiva
\item Respeitar a legislação pertinente a área de TI
\item Aprimorar o monitoramento, a avaliação e a mensuração dos serviços de TI
\end{enumerate}

